\section{はじめに}

哲学は伝統的に経験的知識 empirical knowledge とア・プリオリな知識 a priori knowledge を区別してきた.前者の経験的知識は実験や観察を通して獲得された知識であり,後者のア・プリオリな知識は経験によらず(理性のみによって)獲得された知識である.数学や論理学の知識はア・プリオリな知識であるとされてきた.

一般に,経験的知識がもちうる確実性は数学や論理学が知識がもつ確実性と比べて低く見られがちである.たしかに一旦獲得された数学や論理学の知識がほとんど不可謬であることに対して,経験的知識には常に見落としや錯覚の可能性があり,新しい経験的データに基づく反証の可能性にも開かれている.

他方で,絶対に誤ることはない,と言いたくなるような経験的知識もある.たとえば自分の手をじっと見つめて,「ここに手がある」ことを知る.これは間違いなく確実であるように思われる.そして,これはMooreが「外的世界の証明」\cite{PEW}で実際に示そうとした例でもある.

本稿の目的は,このような命題が確実であるということについて,可能な限り\kenten{正確に}主張することである.まずは2章で,Mooreの証明ではこの命題の確実性を十分に主張できていないということを示す.そして3章で,Wittgensteinの『確実性の問題』\cite{OC}における議論を参照することで,経験命題がもちうる確実性についての分類と分析とを行い,そのうえであらためて主張を試みる.
