\section{Mooreの証明と懐疑論}

まずは問題となる「外的世界の証明」の証明プロセスを以下に示す.
\begin{quote}
  わたしは少し前この机の上にふたつの手を示した.したがって,少し前ふたつの手が存在していた.したがって,過去のある時点では少なくともふたつの外的な事物が存在した.\cite[p. 168]{PEW}
\end{quote}
この証明を素朴に解釈するなら,(1)「手がある」ことを経験的に確かめ,(2)「手がある」ならば「外的な事物が存在する」という隠れた前提とあわせて,(3)「外的な事物が存在する」ことを証明している.

\subsection{懐疑論への応答として}

証明の目的を明らかにするために,Mooreがここで念頭に置いていた外的世界の懐疑論についても確認しておこう.これは古くからある懐疑論の一種で,以下はその現代的なバージョンのひとつである.
\begin{quote}
  ある人(あなた自身と考えてもよい)が邪悪な科学者による手術を受けたと想像せよ.その人の脳(あなたの脳)は身体から取りはずされ,脳を生かしておくための培養液の入った水槽に入れられている.神経の末端は超科学的コンピュータに接続され,そのコンピュータによって,脳の持ち主はすべてがまったく平常通りだという幻想をもたされる.人々も,いろいろな対象も,大空等も,みなあるように思われる.しかし,本当は,その人(あなた)の経験していることはみな,コンピュータから神経末端に伝わる電子的インパルスの結果なのだ.\cite[p. 8]{Chisholm}
\end{quote}
外的世界について知ることができないこのような状態をBIVと呼ぼう.われわれは自らの経験の外側に立つことができないので,結局のところ自分がBIVなのかどうかを知ることはできない.一旦これに同意するとしよう.すると,懐疑論者は次のように続ける.
\begin{enumerate}
\item BIVではないことを知らない(同意した前提)
\item 外的世界について何か知っているなら,BIVではないことを知っている(自明な前提)
\item 外的世界について何も知らない(結論)
\end{enumerate}
このように,懐疑論者の主張に同意すると,modus tollens風の推論によって外的世界に関するあらゆる知識を阻却されてしまう.

modus tollens風の推論と言ったのは,これが単なるmodus tollensではないからである.実際にここで採用されている推論規則は,「Pを知っている」をK(P)のように表記するなら
\[K(P{\to}{\lnot}Q), {\lnot}K({\lnot}Q){\models}{\lnot}K(P)\]
のように書ける.これと,Mooreの証明が採用していた以下のような推論形式とは,対偶をとることで相互に変換が可能である.
\[K(P{\to}{\lnot}Q), K(P){\models}K({\lnot}Q)\]
なお,このような推論規則を認めることは,知識が演繹的推論のもとで閉じた集合であることを意味するため,知識の「閉包原理 closure principle」と呼ばれている\cite{DeRose}.

以上のことから,両者は互いの前提を否定する逆説的な関係にあるものの,推論の妥当性に関しては同等である.つまり,Mooreの証明を経験的証明として解釈したとき,懐疑論の論駁としては成功していない.そして懐疑論の主張が有効であるなら,自分の手があることについて,Mooreは十分に主張できていない.

ではどうすればよいか.ここでふたつの可能性を考えることができる.
\begin{enumerate}
\item この証明はそもそも経験的証明ではない
\item この証明は懐疑論の論駁には成功していないが,正しい主張である
\end{enumerate}
次節ではこのふたつの可能性を見ていく.

\subsection{証明の解釈をめぐって}

AmbroseやMalcolmは,Mooreの証明を経験的証明として解釈することはできないと考えている\cite{Stroud}.というのも,かれらによると,懐疑論者の主張は「経験的な主張」ではないため,それに対して経験的な仕方で応答することは意味をなさない.
\begin{quote}
  懐疑論者は知識の論理的不可能性を論じているのであり,いかなる経験的事実も論じてはいない.\cite[p. 402]{Schilpp}.
\end{quote}
懐疑論者の主張が,われわれは原理的に「知りえない」という純粋に論理的な主張であるなら,いくら経験的事実について「知っている」例を挙げたところで,反論にはならない.

そこで,かれらはMooreの証明が論駁に成功するような別の解釈を提案する.たとえばMalcolmの解釈によると,Mooreの証明は「われわれの言語感覚」に訴えるものである\cite[p. 354]{Schilpp}.すなわち,Mooreが証明で示したことは,現にわれわれはこのような仕方で「知っている」という言葉を用いているという「言語の使用」に関する指摘であり,そもそも経験的証明ではなかった.

しかし,こうした解釈はStroud\cite{Stroud}によって否定されている.Stroudは,Malcolmのような解釈が可能であることを認めつつも\cite[p. 162]{Stroud},Moore自身の記述を根拠として,この証明が明らかに経験的証明として書かれたものであることを示している\cite[p. 158f.]{Stroud}.また,Stroudはかれらの主張に対して次のようにコメントしている.
\begin{quote}
  ムーアの証明についてのこれらの論評は,次のことを説明するために提出されている.つまり,哲学的懐疑論に対する応答として,表面上は不十分に見えるものが,それにもかかわらず哲学的な威力と哲学的な重要性を大いにもちうるのはどのようにしてか,ということである.確かにムーアは,\kenten{一見したところ}マルコムとアンブローズが主張するようなことを行ってはいない.それだからマルコムとアンブローズは,自らの解釈をさらにもっともらしくするために,ムーアが見かけ上行っているようなことに対して直接反論するということをしにかかるのである.
\end{quote}
Stroudの見解では,Mooreの証明が経験的証明ではないというMalcolmやAmbroseの解釈は,あくまで自説を支持するためのものであり,正確な読解ではない.

こうしてMooreの証明を経験的証明として解釈するかわりにStroudが考えている可能性は,この証明がそもそも懐疑論の論駁を意図していないという可能性である.
\begin{quote}
  ムーアははからずも,われわれに次のような可能性を示していると言えるかもしれない.すなわち,外界についてわれわれがもっている知識をめぐる哲学上の問題にたいして,何らかの回答を与えることはしなくとも,日常生活の中でムーアやわれわれが述べたり行なったりしていることは,完全に真でありかつ正当であるという可能性である.\cite[p. 170]{Stroud}
\end{quote}
ここで想定されている可能性は,あらゆる知識を否定する懐疑論と,Mooreの証明のような仕方で知識を主張することとが「両立可能」であるという可能性である.

しかし,この可能性も,最終的にStroud自身が「理解しがたい」として棄却している\cite[p. 212]{Stroud}.Mooreの証明が懐疑論と両立するという選択は,日常的な言明が哲学的懐疑論による攻撃をまったく受け付けないということである.それは,哲学における知識理論と日常的な知識言明とがまったく無関係であるという事態を意味している.このような事態はたしかに理解しがたい.

\subsection{前提の確実性を比較する}

哲学の議論と日常的な言明とを同じ枠組みに捉えたうえで,Mooreの証明が経験的証明であることを認めつつ,それを懐疑論と両立させる方法はあるだろうか.

前提の確実性を比較する,という方法はまだ残っているように思われる.つまり,論証の妥当性において同等であるMooreの証明と懐疑論とを,前提の確実性において比較するのである.David Lewisは次のように述べていた\cite[\S2]{DeRose}.
\begin{quote}
  Mooreの事実……それは,対立するどんな哲学的議論の前提についてわれわれが知っていることよりも,さらによく知っていることのひとつである.
\end{quote}
われわれにとって,Mooreの前提は懐疑論者の前提よりも馴染み深い.われわれが究極的にどちらの主張を信じたいかと訊かれれば,Mooreの主張であると答えたくなる.

だとすると,奇妙な点は,Mooreがこの前提を経験的に証明しようとした点にある.なぜなら,われわれにとってMooreの前提が懐疑論者の前提よりも明らかなことであるなら,Mooreはそれを証明する必要などなかったはずだからである.ここには,確実な事柄を証明するということの不自然さがある.

次章では,Wittgensteinの『確実性の問題』における議論を通して,この不自然さを解明する.
