\section{おわりに}

本稿では,「手がある」という経験命題が確実であることを,可能な限り正確に主張することを試みた.
\begin{quote}
  わたしは哲学者と庭にすわっている.かれは繰り返し「あそこに木があることを知っています」と言い,われわれの近くの木を指差す.ほかのひとが来てそれを聞き,わたしはかれにこう言う.「この男は狂気なのではありません.われわれはただ哲学をやっているんです」.\cite[\S467]{OC}
\end{quote}
われわれの\kenten{哲学}が多くのことを明らかにしたと信じよう.

最後に,調査で明らかにしたことをまとめておく.(1)知識と確実性とは異なるカテゴリーに属する.(2)確実性には探究の方向を規定するための規範としての機能と,より複雑な事柄の正誤を判定するための物差しとしての機能とがある.そして,物差しが正しくないということはありえない.これは論理的な確実性である.(3)論理的な確実性をもつ命題も,経験的な確実性をもたないことがある.その場合,その命題を経験的に探究することは可能である.しかしそのことによって論理的な確実性が揺らぐことはない.(4)すべてを疑おうとする懐疑論でさえ,何らかの確実性を前提している.その確実性は命題ではなくむしろ行動様式であり,それが実践である限り排除できないものである.
