\section{Wittgensteinにおける確実性}

\subsection{知識と確実性を区別する}

通常,「信じる」や「知る」,「疑う」などの用語を有意味なものとして用いるためには,「知るにいたる get to know」\cite[p. 24]{BB}プロセスが存在する必要がある.すなわち,あらかじめ自明な事柄については,そもそもこれらの用語を適用する余地がない.たとえば,「わたしは痛みをもっている」ということを,わたし自身が「知るにいたる」ことはない.よって,「わたしは痛みをもっていることを知っている」という表現は不自然な表現となる.この表現は,「わたしは……知っている」の部分が意味をなしていない.このように,「知るにいたる」プロセスが存在しない命題は,探究の対象から外れるがゆえに,探究に関する諸用語も適用されない.

では,「手がある」はどうだろうか.もし「手がある」が「わたしは痛みをもっている」のように自明な命題であれば,これを知識として扱うことは不自然である.しかし,次のように「手がある」かどうかを実際に探究する場面を考えることは可能である.
\begin{quote}
  たとえば,手に大怪我をした際の手術後にまだ麻酔が効いているような状態で「私の手がある」ということを私が疑うということはありうるであろう.\cite[pp. 53f.]{Yamada}
\end{quote}
このような特殊な状況を想定するなら,「手がある」という命題が経験的に探究されることはありうる.

しかしながら,このような特殊な状況ではなく,あくまで通常の状況を想定するなら,やはり「手がある」ことは確実であるように思われる.通常の状況においては「手がある」ことを誤るということはありえないし,そのことをMooreが試みたような仕方であらためて証明する必要もない.
\begin{quote}
  わたしに両手があるということは,両手を見つめる前にすでに確実なことであり,見つめたからといって確実性にかわりがあるわけではない.にもかかわらず,「わたしに両手がある,というのは絶対に揺るがぬ信念である」と言って差し支えないはずだ.それは何事であれ,わたしはこの命題にたいする反証とみなすつもりではない,ということを表現するであろう.\cite[\S245]{OC}
\end{quote}
通常の状況では,わたしは自分自身に手があることを経験的に探究するわけでもなければ,「知っている」わけでもない.したがって,そのことへ向けられた疑いもまた意味をなさない.それは,わたしにとって最も揺るぎない事実のひとつであり,探究すべき事柄には属していない.

このように,確実性は探究の対象にならないという点において,「「知識」と「確実性」とは異なる\kenten{カテゴリー}に属する」\cite[\S308]{OC}.知識は言語ゲームの内部で公共的に探究されうる概念であるが,確実な事柄は「……にとって……は確実である」と表現されるように,非公共的な概念である.
\begin{quote}
  Mooreは「わたしは……を知っている」というかわりに,「わたしにとって……は確実である」と,さらに「ほかのひとびとにとってもそれは確実である」と言ってもよかったのではないか.\cite[\S116]{OC}
\end{quote}
通常,「手がある」という命題は確実性のカテゴリーに属しており,知識ではない.この点が理解されていなかったことは,おそらくMooreの誤謬のひとつである.

\subsection{探究の規範として}

確実な事柄が探究の対象ではないのだとすると,それはいったいどのような意味をもつのだろうか.

例として「$2+2=4$」という数学的命題の応用について考えてみよう\cite[\S3 1-1]{Yamada}.われわれは「$2+2=4$」という命題を根拠に「2個のりんごと2個のりんごをあわせると4個になる」という事態を想定している.ところが,実際に試してみた結果3個にしかならなかった.このとき「$2+2=4$」を疑うことはせず,かわりに「1個がどこかになくなったに違いない」などと考える.つまり,ここでは「$2+2=4$」への疑いが排除されることで,この場面で真に探究すべき対象が明らかになっている.このように,「$2+2=4$」はこの場面において「「疑いをどこに向けて,間違いをどこに探すべきか」という探究の方向性を規定する規則としての役割」\cite[p. 169]{Yamada}を果たしている.

経験命題の場合もこれと同様である.たとえば科学の実験において「目の前の実験器具の存在を疑うことはしない.わたしが疑うことはいくらでもあるだろうが,\kenten{それ}を疑うことはしない」\cite[\S337]{OC}.もちろん理由は,実験器具の存在を疑っているようでは,真に実験すべき対象を実験することができないからである.すなわち,「実験器具が存在する」という経験命題を\kenten{疑ってはならない}ということは,実験を行うための規範に属する.この意味において,この経験命題は先ほどの「$2+2=4$」と同等の地位を有している.

このように,ある命題を確実だと見なすことは,探究すべき方向を規定するということでもある.このとき,確実な命題は探究の規範としての役割をもっている.この役割を,Wittgensteinは次のような蝶番の比喩で表現している.
\begin{quote}
  すなわち,われわれが立てる\kenten{問い}と\kenten{疑い}とは,ある命題が疑いから免除され,その問いや疑いを回転させる蝶番 hinge のようであることによって成立している.\cite[\S341]{OC}
\end{quote}
蝶番を固定することによって始めて一定の方向へ回転させることができるように,確実な命題を固定することによって始めて方向づけられた探究が可能になる.

では,固定されて探究の規範となる命題は,どのような種類の命題なのだろうか.この点についてのWittgensteinの見解は次のようなものである.
\begin{quote}
  たしかにわたしは以下のように言いたい.いかなる経験命題も公準に変換されうる,そして記述の規範となる,と.しかしわたしはなおこのことに対して信用しかねる.この命題はあまりに一般的に過ぎる.人はほとんどこう言いたくなるだろう.「どんな命題も理論的には……変換されうる」しかし,ここで「理論的」とは何を意味するのか.それはあまりにも『論理哲学論考』のように響く.\cite[\S321]{OC}
\end{quote}
あらゆる経験命題が規範となりうる,もしこれが正しければ,たとえば「$2+2=3$」が計算の規範となることも可能である.それはちょうど,チェスのルールを恣意的に変更することが可能であるように,計算の規範もまた恣意的に変更することが可能であるということである.しかし,こうした考えは「計算」をゲームとしてあまりにも理想化しすぎている.

\subsection{実践と基本形式}

一方で,山田\cite{Yamada}が指摘しているように,「計算」は単に理想化されたゲームではなく,現実にわれわれが行ってきた実践的行為の一種である.そして,このような観点から捉えることで,規範の選択に関する非恣意的な側面が明らかになる.
\begin{quote}
  「$12{\times}12$」のような初歩的な計算に対して「$144$」という答えを与える行動パターンを,ウィトゲンシュタインの言葉を借りて計算ゲームの「基本形式 (Grundform)」と呼ぶとすれば,われわれの計算ゲームの基本形式は数えられないほどの偶然的で自然的な事実に支えられて成立している.\cite[p. 174]{Yamada}
\end{quote}
ここで言われている「基本形式」とは,より複雑な計算の正誤を判定する「物差し」\cite[\S492]{OC}であり,「足場」\cite[\S211]{OC}である\cite[p. 174]{Yamada}.そしてそれは,われわれが常にそのような仕方で実践を行ってきたという多くの「自然的事実」に支えられている.

このように,現実の実践的行為において規範となっているものを調べていけば,現にわれわれはそのように行為してきた,という多くの自然的事実に行き当たる.そうしたサポートがある以上,あたかもチェスのルールを変更するかのように,現実の実践における規範を恣意的に変更することはできない.

\subsection{Moore命題の確実性}

以上の議論を踏まえて再びMooreによる証明の前提(Moore命題)を考察する.

山田によると,手をじっと見つめて「ここに手がある」という判断を下すことは,「知覚の信頼性」を蝶番とする「判断の基本形式」である\cite[p. 178f.]{Yamada}.このような基本形式は,命題というよりは,むしろそのように振る舞うという行動様式である.すなわち,Moore命題がもつ確実性の背後には,われわれは知覚への信頼を含むような仕方で実際に振る舞ってきた,という数々の自然的事実によるサポートが存在する.

そして,このような「判断の基本形式」はわれわれのあらゆる実践の基本形式,すなわち「言語ゲーム全体の蝶番」となっている\cite[p. 180]{Yamada}.したがって,これを疑うことは言語ゲーム全体への疑いを含意する.

このように,Moore命題はあらゆる実践の基本形式としての確実性を有している.ただし,この確実性はあくまでも基本形式としての「論理的な確実性」であり,経験的な探究の末に獲得された「経験的な確実性」ではない.
\begin{quote}
  つまり,およそ正誤を測るための物差しがそれ自体間違いでありえないということは,世界のあり方を探究した結果として獲得された経験的な確実性を意味しているのではなく,「間違いが\kenten{論理的}に排除されている」という意味での論理的な確実性を意味しているのである.\cite[p. 181]{Yamada}
\end{quote}
Moore命題が経験的な確実性をもたない以上,この命題を経験的に探究することもまた可能である.しかし,それは基本形式としての「論理的な確実性」を揺るがすものではない.

\subsection{懐疑論に対して}

以上の結論は,Mooreによる証明の前提が確実であることと,哲学的懐疑論の議論とを,同じ枠組みにおいて説明している.したがって,Stroudが懸念していた事態は起こらない.すなわち,懐疑論者はプレイしているゲームが異なっていただけである.われわれは懐疑論というゲームをプレイすることができるが,それはもはや通常の言語ゲームではない.それはちょうど,チェスのゲーム性の中に,ピースがひとりでに動くのではないかという疑いを含めないことと同じである.われわれはそれを疑うことができるが,その場合,それはもはやチェスをプレイしているわけではない.

そして,懐疑もまたひとつの実践であり,言語ゲームである以上,そこには必ず何らかの蝶番が固定されている.
\begin{quote}
  すべてを疑おうとする人は,疑うところまで行き着くことすらできないだろう.疑いのゲームは,すでに確実性を前提としているのだ.\cite[\S115]{OC}
\end{quote}
あらゆる知識を疑う懐疑論もまた,判断がつねに一定であるという確実性を前提とする実践のうえに成り立っているということは,もはや言うまでもないことである.
